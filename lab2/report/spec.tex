\documentclass[12pt, a4paper]{article}
\usepackage[a4paper, includeheadfoot, mag=1000, left=2cm, right=1.5cm, top=1.5cm, bottom=1.5cm, headsep=0.8cm, footskip=0.8cm]{geometry}
% Fonts
\usepackage{fontspec, unicode-math}
\setmainfont[Ligatures=TeX]{CMU Serif}
\setmonofont{CMU Typewriter Text}
\usepackage[english, russian]{babel}
% Indent first paragraph
\usepackage{indentfirst}
\setlength{\parskip}{5pt}
% Page headings
\usepackage{fancyhdr}
% Diagrams
\usepackage{graphicx}
% Code listing
\usepackage{listings}
\pagestyle{fancy}
\renewcommand{\headrulewidth}{0pt}
\setlength{\headheight}{16pt}
%\newfontfamily\namefont[Scale=1.2]{Gloria Hallelujah}
\fancyhead{}
% Use case template

\begin{document}

% Title page
\begin{titlepage}
\begin{center}

\textsc{Университет ИТМО}
\vfill
\textbf{Лабораторная работа №3\\[4mm]
Вычислительная математика\\[16mm]
Интерполяция функций}\\[16mm]
\begin{flushright}
~\\[2mm]Саржевский Иван
~\\[2mm]Группа P3202
~\\[2mm]Вариант 13
\end{flushright}
\vfill
Санкт-Петербург\\[2mm]
2018

\end{center}
\end{titlepage}


\section*{Цель}
Написать программу для поиска функции, являющейся наилучшим приближением заданной
табличной функции по методу наименьших квадратов.

\section*{Порядок выполнения}
\begin{itemize}
  \item Вычислить меру отклонения S для каждой из исследуемых функций
  \item Уточнить значения коэффициентов эмпирических функций a, b и c
  \item Сформировать массивы предполагаемых эмпирических зависимостей
  \item Определить среднеквадратичное отклонение для каждой из исследуемых
    функций и выбрать самую эффективную
  \item Построить графики полученных эмпирических функций
\end{itemize}

\subsection*{Функция} 
\begin{center}
  \begin{tabular}{c c c c c c c c c c c c}
    x & 2.0 & 2.7 & 3.4 & 4.1 & 4.8 & 5.5 & 6.2 & 6.9 & 7.6 & 8.3 & 9.0 \\
    y & 0.22 & 1.11 & 2.23 & 3.34 & 4.45 & 5.27 & 6.41 & 7.34 & 8.47 & 9.5 & 10.23
  \end{tabular}
\end{center}
\section*{Рабочие формулы методов}
\[S = \sum_{i=1}^n (\varphi(x_i) - y_i)^2\] 
\[\delta = \sqrt{\frac{\sum_{i=1}^n (\varphi(x_i) - y_i)^2}{n}}\]
Для линейной аппроксимации:
\[a = \frac{n\sum x_i y_i - \sum x_i \cdot \sum y_i}{n\sum x_i^2 - \sum x_i \cdot \sum x_i}, b = \frac{\sum x_i^2 \sum y_i - \sum x_i \sum x_i y_i}{n\sum x_i^2 - \sum x_i \cdot \sum x_i}\]
Для квадратичной аппроксимации:
\begin{center}
  \includegraphics[scale = 0.5]{square_appr_formula}
\end{center}
Для применения метода наименьших квадратов экспоненциальная функция линеаризуется:
\[ln(a^{ebx}) = ln(a) + bx\]
Получим линейную зависимость $Y = AX + B$, после определения коэффициентов A и B
вернемся к изначальным определениям: $a = e^A, b = B$.

Как и в случае экспоненциальной аппроксимации, для $\varphi(x) = a + bln(x)$ 
вводятся обозначения $Y = y, X = ln(x), A = a, B = b$. После определения
линейных коэффициентов восстановим $a = A$ и $b = B$.

Введем для степенной аппроксимации $\varphi(x) = ax^b$ обозначения $Y = ln(\varphi(x)), X = ln(x), A = ln(a), B = b$,
найдем линейные коэффициенты $A$ и $B$, выразим $a$ и $b$ обратно: $a = e^A, b = B$.

\section*{Реализация на языке Python}

\begin{lstlisting}[language = Python]
def linear_approximation(xs, ys):
    a = (sxy(xs, ys)*len(xs) - sx(xs)*sx(ys))/(sxx(xs)*len(xs) - sx(xs)**2)
    b = (sxx(xs)*sx(ys) - sx(xs)*sxy(xs, ys))/(sxx(xs)*len(xs) - sx(xs)**2)
    return [a, b]

# a3, a2, a1
def square_approximation(xs, ys):
    left = np.array([
                     [len(xs), sx(xs), sxx(xs)],
                     [sx(xs), sxx(xs), pow_n(xs, 3)],
                     [sxx(xs), pow_n(xs, 3), pow_n(xs, 4)]
                    ])
    right = np.array([sx(ys), sxy(xs, ys), sxxy(xs, ys)])
    return np.linalg.solve(left, right)

def exponent_approximation(xs, ys):
    new_y = np.array([])
    for i in ys:
        new_y = np.append(new_y, [math.log(i, math.exp(1))])
    res = linear_approximation(xs, new_y)
    res[0] = math.exp(res[0])
    return res

def power_approximation(xs, ys):
    new_y = np.array([])
    for i in ys:
        new_y = np.append(new_y, [math.log(i, math.exp(1))])
    new_x = np.array([])
    for i in xs:
        new_x = np.append(new_x, [math.log(i, math.exp(1))])
    res = linear_approximation(new_x, new_y)
    res[0] = math.exp(res[0])
    return res

def logarithmic_approximation(xs, ys):
    new_x = np.array([])
    for i in xs:
        new_x = np.append(new_x, [math.log(i, math.exp(1))])
    return linear_approximation(new_x, ys)
\end{lstlisting}

\section*{Вывод программы для заданной функции}
\subsection*{Линейная аппроксимация}
\includegraphics[scale = 0.7]{linear}
\subsection*{Квадратичная аппроксимация}
\includegraphics[scale = 0.7]{square}
\subsection*{Экспоненциальная аппроксимация}
\includegraphics[scale = 0.7]{exponent}
\subsection*{Логарифмическая аппроксимация}
\includegraphics[scale = 0.7]{logarithm}
\subsection*{Степенная аппроксимация}
\includegraphics[scale = 0.7]{power}
\subsection*{Анализ результатов работы}
\includegraphics[scale = 0.7]{result_table}

\section*{Вывод}
В ходе проделанной работы были вычислены меры отклонения для линейной,
квадратичной, экспоненциальной, степенной и логарифмической функций.
По методу наименьших квадратов было выявлено, что наилучшим приближением для
заданной табличной функции является квадратичная функция.

\end{document}
