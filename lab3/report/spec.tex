\documentclass[12pt, a4paper]{article}
\usepackage[a4paper, includeheadfoot, mag=1000, left=2cm, right=1.5cm, top=1.5cm, bottom=1.5cm, headsep=0.8cm, footskip=0.8cm]{geometry}
% Fonts
\usepackage{fontspec, unicode-math}
\setmainfont[Ligatures=TeX]{CMU Serif}
\setmonofont{CMU Typewriter Text}
\usepackage[english, russian]{babel}
% Indent first paragraph
\usepackage{indentfirst}
\setlength{\parskip}{5pt}
% Page headings
\usepackage{fancyhdr}
% Diagrams
\usepackage{graphicx}
% Code listing
\usepackage{listings}
\pagestyle{fancy}
\renewcommand{\headrulewidth}{0pt}
\setlength{\headheight}{16pt}
%\newfontfamily\namefont[Scale=1.2]{Gloria Hallelujah}
\fancyhead{}
% Use case template

\begin{document}

% Title page
\begin{titlepage}
\begin{center}

\textsc{Университет ИТМО}
\vfill
\textbf{Лабораторная работа №3\\[4mm]
Вычислительная математика\\[16mm]
Интерполяция функций}\\[16mm]
\begin{flushright}
~\\[2mm]Саржевский Иван
~\\[2mm]Группа P3202
~\\[2mm]Вариант 13
\end{flushright}
\vfill
Санкт-Петербург\\[2mm]
2018

\end{center}
\end{titlepage}


\section*{Цели и задачи}
Решить задачу интерполяции: найти значения функции при заданных значениях
аргумента, отличных от узловых точек.

Для исследования использовать:
\begin{itemize}
  \item линейную и квадратичную интерполяцию;
  \item многочлен Лагранжа;
  \item многочлен Ньютона.
\end{itemize}

\section*{Рабочие формулы}
\subsection*{Линейная интерполяция}
Уравнения каждого отрезка ломаной линии в общем случае разные. Поскольку
имеется n интервалов, то для  каждого из них в качестве  уравнения
интерполяционного полинома используется уравнение прямой, проходящей через
две точки. В частности, для i-го интервала можно написать уравнение прямой,
проходящей через точки $(x_{i-1}, y_{i-1})$ и $(x_i, y_i)$ в виде:
\[\frac{y - y_{i-1}}{y_i - y_{i-1}} = \frac{x - x_{i-1}}{x_i - x_{i-1}}\]
Отсюда: $y = a_ix + b_i$ на промежутке $[x_{i-1}, x_i]$
$$a_i = \frac{y_i - y_{y-1}}{x_i - x_{i-1}} (1), b_i = y_{i-1} - a_ix_{i-1} (2)$$

\subsection*{Квадратичная интерполяция}
В качестве интерполяционной функции принимается квадратный трехчлен на отрезке
$[x_{i-1}, x_{i+1}]$

Неизвестные коэффициетны a, b и c находим из условий прохождения параболы
через три данные точки:
$$a_ix_{i-1}^2 + b_ix_{i-1} + c_i = y_{i-1}$$
$$a_ix_i^2 + b_ix_i + c_i = y_i$$
$$a_ix_{i+1}^2 + b_ix_{i+1} + c_i = y_{i+1} (3)$$

\subsection*{Многочлен Лагранжа}
Неизвестные значения функции $f(x) = y$ вычисляются как значения
интерполяционного многочлена Лагранжа $Ln(x)$, который принимает данные
значения $y_0, ..., y_i$ в наборе точек $x_0, ..., x_i$.
$$Ln(x) = \sum_{i=0}^{n-1} y_i \prod_{j=0,\ j\neq i}^{n-1} \frac{x - x_j}{x_i - x_j}$$

\subsection*{Интерполяционные формулы Ньютона для равноотстоящих узлов}
Если узлы интерполяции равноотстоящие и упорядочены по величине, то есть
$x_{i+1} - x_i = h = const$, то интерполяционный многочлен может быть записан
в форме Ньютона.

Прямая формула:
$$Nn(x) = y_i + t\Delta y_i + \frac{t(t-1)}{2!}\Delta^2 y_i + ... + \frac{\prod_{j=0}^{n-1} (t-j)}{n!}\Delta^n y_i,\ t = \frac{x - x_i}{h}$$
Обратная формула:
$$Nn(x) = y_n + t\Delta y_{n-1} + \frac{t(t+1)}{2!}\Delta^2 y_{n-2} + ... + \frac{\prod_{j=0}^{n-1} (t+j)}{n!}\Delta^n y_0,\ t = \frac{x - x_n}{h}$$
Где $\Delta^k y$ - конечная разность $k$-го порядка

Конечная разность первого порядка равна $\Delta y_i = y_{i+1} - y_i$

Конечные разности высших порядков выражаются как:
$$\Delta^k y_i = \Delta^{k-1} y_{i+1} - \Delta^{k-1} y_i$$
\subsection*{Интерполяционная формула Ньютона для неравноотстоящих узлов}
$$Nn(x) = f(x_0) + \sum_{k=1}^n f(x_0, x_1, ..., x_k) \prod_{j=0}^{k-1} (x - x_j)$$
Где $f(x_0, ..., x_k)$ — разделенные разности $k$-порядка.
Разделенные разности нулевого порядка равны значению функции в узле,
то есть $f(x_i) = y_i$; разности высших порядков выражаются как:
$$f(x_i, x_{i+1}, ..., x_{i+k}) = \frac{f(x_{i+1}, ..., x_{i+k}) - f(x_i, ..., x_{i+k-1})}{x_{i+k} - x_i}$$

\section*{Вычисление значения функции в заданной точке методами линейной
  и квадратичной интерполяции}
\subsection*{Линейная интерполяция}
$x = 0.312$. Табличное задание функции - см.табл.1

Определим интервал, в который входит этот x.
Интервал: $[0.31, 0.317]$

Соответствующие значения функции: $3.1218, 3.0482$.
Определим коэффициенты a и b по формулам (1) и (2)

$a = -10.51428571428569, b = 6.381228571428563$

Найдем значение функции $a*x+b$ в точке x

$f(x) =  3.100771428571428$

\subsection*{Квадратичная интерполяция}
$x = 0.312$. Табличное задание функции - см.табл.1

Определим интервал, в который входит этот x.
Интервал: $[0.31, 0.317, 0.323]$

Соответствующие значения функции: $3.1218, 3.0482, 2.9875$.
Составим систему уравнений(3) и решим её

$a = 30.58608058608058, b = -29.691758241758222, c = 9.386922710622704$

Найдем значение функции $a*x^2 + bx + c$ в точке x

$f(x) =  3.100465567765567$

Полученный график можно увидеть в Приложении 1

\section*{Листинг программы}
\begin{lstlisting}[language = Python]
  from math import factorial

def linear_interpolation(xs, ys, x):
    x_i = min([d for d in xs if d > x])
    x_i_minus_1 = xs[np.where(xs == x_i)[0][0] - 1]
    y_i = ys[np.where(xs == x_i)[0][0]]
    y_i_minus_1 = ys[np.where(xs == x_i_minus_1)[0][0]]

    a_i = (y_i - y_i_minus_1) / (x_i - x_i_minus_1)
    b_i = y_i_minus_1 - a_i * x_i_minus_1

    y_x = a_i * x + b_i

    return y_x

def square_interpolation(xs, ys, x):
    x_i = min([d for d in xs if d > x])
    x_i_minus_1 = xs[np.where(xs == x_i)[0][0] - 1]
    x_i_plus_1 = xs[np.where(xs == x_i)[0][0] + 1]

    y_i = ys[np.where(xs == x_i)[0][0]]
    y_i_minus_1 = ys[np.where(xs == x_i_minus_1)[0][0]]
    y_i_plus_1 = ys[np.where(xs == x_i_plus_1)[0][0]]

    c = np.linalg.solve(
        [
            [x_i_minus_1**2, x_i_minus_1, 1],
            [x_i**2, x_i, 1],
            [x_i_plus_1**2, x_i_plus_1, 1]
        ],
        [y_i_minus_1, y_i, y_i_plus_1]
    )

    y_x = c[0] * x**2 + c[1] * x + c[2]

    return y_x

def lagrandge_interpolation(xs, ys, x):
    y_x = 0.0
    for i in range(len(xs)):
        p = 1.0
        for j in range(len(xs)):
            if i != j:
                p *= (x - xs[j]) / (xs[i] - xs[j])
        y_x += p * ys[i]
    return y_x


def diff_difference(xs, ys):
    if len(xs) == 2:
        return (ys[1] - ys[0]) / (xs[1] - xs[0])
    else:
        return (diff_difference(xs[1:], ys[1:]) - diff_difference(xs[:-1], ys[:-1])) / (xs[len(xs) - 1] - xs[0])

def difference(ys, k, i):
    if k == 1:
        return ys[i + 1] - ys[i]
    else:
        return difference(ys, k - 1, i + 1) - difference(ys, k - 1, i)

def newton_interpolation_forward(xs, ys, x):
    h = xs[1] - xs[0]
    x_i_index = np.where(xs == min([d for d in xs if d > x]))[0][0]
    res = 0.0
    res += ys[x_i_index] # x = y0
    t = (x - xs[x_i_index]) / h
    res += t * difference(ys, 1, x_i_index) # x = y0 + t * diff1(y0)
    for i in range(len(xs) - x_i_index - 2):
        t = t * (t - i + 1)
        res += (t / factorial(i + 2)) * difference(ys, i + 2, x_i_index)
    return res

def newton_interpolation_backward(xs, ys, x):
    h = xs[1] - xs[0]
    x_i_index = np.where(xs == min([d for d in xs if d > x]))[0][0]
    res = 0.0
    res += ys[len(ys) - 1] # x = y0
    t = (x - xs[len(xs) - 1]) / h # t = (x - x_n) / h
    res += t * difference(ys, 1, len(ys) - 2) # x = y0 + t * diff1(y0)
    for i in range(len(xs) - x_i_index - 2):
        t = t * (t + i - 1)
        res += (t / factorial(i + 2)) * difference(ys, i + 2, ys[len(ys) - i + 2])
    return res

def newton_interpolation_diff(xs, ys, x):
    x_i = min([d for d in xs if d > x])
    res = ys[np.where(xs == x_i)[0][0]]
    for i in range(np.where(xs == x_i)[0][0], len(xs) - 1):
        p = 1.0
        xss = xs[np.where(xs == x_i)[0][0]:i + 2]
        for j in xss[:-1]:
            p *= (x - j)
        p *= diff_difference(xss, ys[np.where(xs == x_i)[0][0]:i + 2])
        res += p
    return res
\end{lstlisting}

\section*{Результаты работы программы}

Таблица 1

\includegraphics{tableone}

--$x_1 = 0.312$--

Линейная: 3.100771428571428

Квадратичная: 3.100465567765567

Лагранж: 3.1022054525799443

Ньютон вперед: 3.1089

Ньютон для неравностоящих: 3.100682142857144

--$x_4 = 0.304$--

Линейная: 3.1686

Квадратичная: 3.1697632653061234

Лаграндж: 3.1685564333963003

Ньютон для неравностоящих: 3.186956321839078

Ньютон вперед: 3.2114471650559997

Таблица 2

\includegraphics{tabletwo}

--$x_2 = 0.261$--

Линейная: 1.4582540000000002

Квадратичная: 1.4561347400000009

Лаграндж: 1.088268094932948

Ньютон вперед: 1.4522466969654722

--$x_3 = 0.534$--

Линейная: 7.5426519999999995

Квадратичная: 7.5443492800000005

Лаграндж: 6.939942122243442

Ньютон назад: 7.5426519999999995

\section*{Вывод}

В ходе выполнения данной лабораторной работы я познакомился с основными
интерполяционными методами

\pagebreak

\section*{Приложение 1. Графики}

\includegraphics[scale = 0.4]{graphic1}

\includegraphics[scale = 0.4]{graphic2}

\end{document}
