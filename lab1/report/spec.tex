\documentclass[12pt, a4paper]{article}
\usepackage[a4paper, includeheadfoot, mag=1000, left=2cm, right=1.5cm, top=1.5cm, bottom=1.5cm, headsep=0.8cm, footskip=0.8cm]{geometry}
% Fonts
\usepackage{fontspec, unicode-math}
\setmainfont[Ligatures=TeX]{CMU Serif}
\setmonofont{CMU Typewriter Text}
\usepackage[english, russian]{babel}
% Indent first paragraph
\usepackage{indentfirst}
\setlength{\parskip}{5pt}
% Page headings
\usepackage{fancyhdr}
% Diagrams
\usepackage{graphicx}
% Code listing
\usepackage{listings}
\pagestyle{fancy}
\renewcommand{\headrulewidth}{0pt}
\setlength{\headheight}{16pt}
\newfontfamily\namefont[Scale=1.2]{Gloria Hallelujah}
\fancyhead{}
% Use case template

\begin{document}

% Title page
\begin{titlepage}
\begin{center}

\textsc{Университет ИТМО}
\vfill
\textbf{Лабораторная работа №3\\[4mm]
Вычислительная математика\\[16mm]
Интерполяция функций}\\[16mm]
\begin{flushright}
~\\[2mm]Саржевский Иван
~\\[2mm]Группа P3202
~\\[2mm]Вариант 13
\end{flushright}
\vfill
Санкт-Петербург\\[2mm]
2018

\end{center}
\end{titlepage}


\section*{Цель}
Написать программу для реализации поиска корней нелинейных уравнений методами половинного деления и Ньютона

\section*{Порядок выполнения}
Построить график данной функции, таблицы работы методов для нахождения корней кубического уравнения. 
\subsection*{Функция} 
$x^3 + 4.81 * x^2 - 17.37 * x + 5.38$
\subsection*{Корни}
$x = -7.29292; x = 0.34506; x = 2.1378$

\section*{Рабочие формулы методов}
\subsection*{Половинного деления}
$$ x_i = \frac{a_i + b_i}{2} $$
\subsection*{Ньютона}
$$ x_n = x_{n-1} - \frac{f(x_{n-1})}{f`(x_{n-1})} $$

\section*{Реализация на языке Python}
\begin{lstlisting}[language=Python]
def half(a, b, eps, i = 1, iter_meta = [["i", "a", "b", "x", "f(a)", "f(b)", "f(x)", "len"]]):
  x_0 = (a + b)/2

  l = interval_len(a, b)

  iter_meta.append([i,round(a, 3),round(b, 3),round(x_0, 3),round(fun(a), 3),round(fun(b), 3),round(fun(x_0), 3),round(l, 3)])

  if abs(fun(x_0)) <= eps:
    return [x_0, fun(x_0), i, iter_meta]

  if validate(a, x_0):
    return half(a, x_0, eps, i + 1, iter_meta)
  else:
    return half(x_0, b, eps, i + 1, iter_meta)

def newton(x_0, eps, iter_meta = []):
    i = 1
    if (abs(derivative(x_0)) < eps):
        return None
    x = x_0 - (fun(x_0)/derivative(x_0))
    while (abs(fun(x_0)) > eps):
        i += 1
        x = x_0 - (fun(x_0)/derivative(x_0))
        iter_meta.append([i, round(x_0, 3), round(fun(x_0), 3), round(derivative(x_0), 3), round(x, 3), round(abs(x_0 - x), 3)])
        x_0 = x
    return [x_0, fun(x_0), iter_meta, i]

def fun(x, coef_0 = 1, coef_1 = 4.81, coef_2 = -17.37, coef_3 = 5.38):
    return (coef_0 * x**3) + (coef_1 * x**2) + (coef_2 * x) + coef_3

def validate(a, b):
    if fun(a)*fun(b) < 0:
        return True
    else: return False

def interval_len(a, b):
    if a < 0 and b > 0:
        l = b + abs(a)
    elif a < 0 and b < 0:
        l = abs(a) - abs(b)
    else: l = b - a
    return l

def derivative(x, coef_0 = 1, coef_1 = 4.81, coef_2 = -17.37):
    return fun(x, 0, 3*coef_0, 2*coef_1, coef_2)

def second_deriative(x, coef_0 = 1, coef_1 = 4.81, coef_2 = -17.37):
    return fun(x, 0, 0, 6*coef_0, 2*coef_1)
\end{lstlisting}

\section*{График функции}
\includegraphics{plot.png}

\section*{Вывод программы}
\subsection*{Первый корень}

1. Half dividing: x = -7.29296875, f(x) = -0.0037679935, 8 iterations

2. Newton: x = -7.292917437647474, f(x) = -0.000071813616382, 4 iterations

\includegraphics{first_root.png}

\subsection*{Второй корень}

1. Half dividing: x = 0.345703125, f(x) = -0.0087018463, 9 iterations

2. Newton: x = 0.34506238308748477, f(x) = 0.000069622187341, 5 iterations

\includegraphics{second_root.png}

\subsection*{Третий корень}

1. Half dividing: x = 2.1376953125, f(x) = -0.0025977225, 10 iterations

2. Newton: x = 2.137995782478928, f(x) = 0.002482390999789, 3 iterations

\includegraphics{third_root.png}

\section*{Вывод}
В ходе выполнения данной лабораторной работы были реализованы методы половинного деления и Ньютона и обнаружены следующие недостатки и 
достоинства каждого из них:
\begin{itemize}
  \item Половинного деления:
  \begin{itemize}
    \item Линейная сходимость
    \item Простая рабочая формула
  \end{itemize}
  \item Ньютона
  \begin{itemize}
    \item Квадратичная сходимость
    \item Вычисление производной для каждой итерации
    \item Чувствительность к выбору первого приближения
  \end{itemize}
\end{itemize}
\end{document}
